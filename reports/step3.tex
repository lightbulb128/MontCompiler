\documentclass[UTF8]{ctexart}
\usepackage{graphicx}
\usepackage{amssymb}
\usepackage{amsmath}
\usepackage{subfigure}
\usepackage{geometry}
\usepackage{caption}
\usepackage{diagbox}
\usepackage{bm}
\usepackage{color}

\usepackage{listings}
\lstdefinestyle{lfonts}{
    basicstyle = \ttfamily
}

\newcommand{\T}[1]{\texttt{{#1}}}
\title{Step 3 Report}
\author{刘轩奇 2018011025}
\date{2020年10月5日}
\geometry{left=2.0cm, right=2.0cm, top=2.5cm, bottom=2.5cm}
\begin{document}
    \maketitle
    \section{工作内容}
        本人选择不使用辅助工具 ANTLR 因此自己实现了 lexer 和 parser。
        \subsection{文件说明} 
            \T{montLexer.h/cpp} 词法分析器;

            \T{montParser.h/cpp} 语法分析器;

            \T{montConceiver.h/cpp} 产生中间代码;

            \T{montAssembler.h/cpp} 从中间代码产生汇编代码;
            
            \T{montLog.h} 记录编译错误信息;

            \T{montCompiler.cpp} MiniDecaf 编译器,包含主函数,编译成功返回 0 否则返回 -1,
            并将错误信息输出到 \T{std::cerr}。
        
        \subsection{本步骤完成的工作}

            \paragraph{1 词法分析} 添加了五种 Token,类型分别为 \T{Plus} 加号,
            \T{Asterisk} 星号即乘号, \T{LSlash} 左斜杠即除号和 \T{Percent} 百分号即取模符号。

            \paragraph{2 句法分析} 添加了 \T{additive}, \T{multiplicative}, \T{primary} 
            类型的树节点。其生成逻辑按照指示书上给出的指导生成。即 \T{additive} 不断生成新的 
            \T{multiplicative} 直到不再出现加减号; \T{multiplicative} 不断生成 \T{unary} 
            直到不再出现乘除模号; \T{primary} 检测括号和单个值。

            \paragraph{3 产生中间代码} 按照指导书上的说明,将对应符号分别生成对应的
            中间代码:\T{ADD}, \T{SUB}, \T{MUL}, \T{DIV}, \T{REM}。

            \paragraph{4 产生汇编代码} 按照指导书的
            说明,将对应中间代码转换为对应汇编语句,使用已有的汇编器编译得到的结果在
            一些操作上并不能看出使用了哪些汇编命令,例如乘法的汇编是调用了一个内置函数。
            于是参考了C++参考实现中的代码,了解到这五种运算对应的操作实际上就是五种
            汇编命令 \T{add, sub, mul, div, rem}。
        
    \section{思考题}
        \paragraph{1} 请给出将寄存器 \T{t0} 中的数值压入栈中所需的 riscv 汇编指令序列;
        请给出将栈顶的数值弹出到寄存器 \T{t0} 中所需的 riscv 汇编指令序列。
        \paragraph{答} \T{addi sp, sp, -4; sw t0, 0(sp);}
        \paragraph{2} 语义规范中规定“除以零、模零都是未定义行为”,但是即使除法的右操作数不是 0,
        仍然可能存在未定义行为。请问这时除法的左操作数和右操作数分别是什么?
        请将这时除法的左操作数和右操作数填入下面的代码中,分别在你的电脑
        (请标明你的电脑的架构,比如 x86-64 或 ARM)中和 RISCV-32 的 
        qemu 模拟器中编译运行下面的代码,并给出运行结果。(编译时请不要开启任何编译优化)
        
        \noindent\rule{\textwidth}{1pt}
        \begin{lstlisting}[style=lfonts]
#include <stdio.h>
int main() {
  int a = Left operand;
  int b = Right operand;
  printf("%d\n", a / b);
  return 0;
}
        \end{lstlisting}
        \noindent\rule{\textwidth}{1pt}

        \paragraph{答} 只要使得算术溢出即可。取 \T{a = 0x80000000; b = -1;} 
        在本机(x86-64 Windows 10)上运行,进程将阻塞,不会给出任何输出。在 QEMU 上执行的
        结果是 \T{0x80000000 = -2147483647}。
\end{document}