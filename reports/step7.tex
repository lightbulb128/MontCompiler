\documentclass[UTF8]{ctexart}
\usepackage{graphicx}
\usepackage{amssymb}
\usepackage{amsmath}
\usepackage{subfigure}
\usepackage{geometry}
\usepackage{caption}
\usepackage{diagbox}
\usepackage{bm}
\usepackage{color}

\usepackage{enumitem}
\setitemize[1]{itemsep=0pt, partopsep=0pt, parsep=\parskip, topsep=5pt}

\usepackage{listings}
\lstdefinestyle{lfonts}{
    basicstyle = \ttfamily
}

\newcommand{\T}[1]{\texttt{{#1}}}
\title{Step 7 Report}
\author{刘轩奇 2018011025}
\date{2020年10月6日}
\geometry{left=2.0cm, right=2.0cm, top=2.5cm, bottom=2.5cm}
\begin{document}
    \maketitle
    \section{工作内容}
        本人选择不使用辅助工具 ANTLR 因此自己实现了 lexer 和 parser。
        \subsection{文件说明} 
            \T{montLexer.h/cpp} 词法分析器;

            \T{montParser.h/cpp} 语法分析器;

            \T{montConceiver.h/cpp} 产生中间代码;

            \T{montAssembler.h/cpp} 从中间代码产生汇编代码;
            
            \T{montLog.h} 记录编译错误信息;

            \T{montCompiler.cpp} MiniDecaf 编译器,包含主函数,编译成功返回 0 否则返回 -1,
            并将错误信息输出到 \T{std::cerr}。
        
        \subsection{本步骤完成的工作}

            什么也没做,就这么用 step6 的代码通过了测试。因为在 step5 中已经做完了有关工作。
        
    \section{思考题}
        \paragraph{1} 请将下述 MiniDecaf 代码中的 \T{???} 替换为一个 32 位整数,使得程序运行结束后会返回 0。
        
        \noindent\rule{\textwidth}{1pt}
        \begin{lstlisting}[style=lfonts]
int main() {
 int x = ???;
 if (x) {
     return x;
 } else {
     int x = 2;
 }
 return x;
}
        \end{lstlisting}
        \noindent\rule{\textwidth}{1pt}

        \paragraph{答} 取 0 即可,此时进入 \T{else} 分支后的声明的变量仅在此作用域中有效,跳出后 \T{x} 仍为初始的 0。

        \paragraph{2} 在实验指导中,我们提到“就 MiniDecaf 而言,名称解析的代码也可以嵌入 IR 生成里”,但不是对于所有语言都
        可以把名称解析嵌入代码生成。试问被编译的语言有什么特征时,名称解析作为单独的一个阶段在 IR 生成之前执行会更好?

        \paragraph{答} 若该语言允许先使用后声明/定义(例如不提前声明而直接相互调用的两个函数),
        或者允许不声明而直接使用(例如 Python 语言,若使用所谓编译器而非解释器),或者允许多文件的声明和定义分离
        (例如 C 的头文件和源文件),则将名称解析单独作为一个阶段更好。
\end{document}