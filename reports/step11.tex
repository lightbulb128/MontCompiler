\documentclass[UTF8]{ctexart}
\usepackage{graphicx}
\usepackage{amssymb}
\usepackage{amsmath}
\usepackage{subfigure}
\usepackage{geometry}
\usepackage{caption}
\usepackage{diagbox}
\usepackage{bm}
\usepackage{color}

\usepackage{enumitem}
\setitemize[1]{itemsep=0pt, partopsep=0pt, parsep=\parskip, topsep=5pt}

\usepackage{listings}
\lstdefinestyle{lfonts}{
    basicstyle = \ttfamily
}

\newcommand{\T}[1]{\texttt{{#1}}}
\title{Step 11 Report}
\author{刘轩奇 2018011025}
\date{2020年10月8日}
\geometry{left=2.0cm, right=2.0cm, top=2.5cm, bottom=2.5cm}
\begin{document}
    \maketitle
    \section{工作内容}
        本人选择不使用辅助工具 ANTLR 因此自己实现了 lexer 和 parser。
        \subsection{文件说明} 
            \T{montLexer.h/cpp} 词法分析器;

            \T{montParser.h/cpp} 语法分析器;

            \T{montConceiver.h/cpp} 产生中间代码;

            \T{montAssembler.h/cpp} 从中间代码产生汇编代码;
            
            \T{montLog.h} 记录编译错误信息;

            \T{montCompiler.cpp} MiniDecaf 编译器,包含主函数,编译成功返回 0 否则返回 -1,
            并将错误信息输出到 \T{std::cerr}。
        
        \subsection{本步骤完成的工作}

            \paragraph{1 词法分析} 没有改变。

            \paragraph{2 句法分析} 按照指导书上的要求修改了以下的树节点:
            \begin{itemize}
                \item[*] \T{unary} 支持解地址和取地址,以及类型转换。
                \item[*] \T{type} 支持指针类型。
                \item[*] \T{assignment} 等号左边的树节点为\T{type}类型。
            \end{itemize}

            \paragraph{3 产生中间代码} 首先添加了类型分析功能,类型附加在每个树节点上,在 parse 阶段不分析
            类型而是在生成中间代码步骤中产生类型。对于一般的控制过程语句(循环、判断节点等)其类型为空(void),
            而表达式类型树节点(包括表达式、各种运算等节点)则生成对应的类型。
            
            此阶段也分析表达式是否为左值。在
            遍历语法树的过程中,根据语法指出当前分析的节点应当是否为左值(例如生成 \T{assignment} 节点
            时,等号左边的对应节点应该为左值),而若产生的实际表达式不是左值则报错。

            同时在生成中间代码的分析过程中
            也分析类型是否正确:函数调用参数类型是否匹配、是否有指针参与数学运算等、
            循环条件表达式是否为指针等等。

            \paragraph{4 产生汇编代码} 没有改变。
        
    \section{思考题}
        \paragraph{1} 为什么类型检查要放到名称解析之后?
        \paragraph{答} 只有名称解析之后才知道函数调用的返回值类型是什么。若不知道函数返回值的类型则无法进行类型检查。
        \paragraph{2} MiniDecaf 中一个值只能有一种类型,但在很多语言中并非如此,请举出一个反例。
        \paragraph{答} 例如 python 中的变量可赋予不类型的值,其具体类型根据赋值而改变,C++中union语法也可以支持
        一个变量中存储的值以不同类型的方式分别处理。
        \paragraph{3} 在本次实验中我们禁止进行指针的比大小运算。请问如果要实现指针大小
        比较需要注意什么问题?可以和原来整数比较的方法一样吗?
        \paragraph{答} 指针比较大小应当注意判断指针的类型是否相同,若指向的类型不相同则无法进行比较,
        若指向的类型相同,则可以比较指向地址的高低,此时等同于无符号整数的比较。
\end{document}