\documentclass[UTF8]{ctexart}
\usepackage{graphicx}
\usepackage{amssymb}
\usepackage{amsmath}
\usepackage{subfigure}
\usepackage{geometry}
\usepackage{caption}
\usepackage{diagbox}
\usepackage{bm}
\usepackage{color}

\usepackage{enumitem}
\setitemize[1]{itemsep=0pt, partopsep=0pt, parsep=\parskip, topsep=5pt}

\usepackage{listings}
\lstdefinestyle{lfonts}{
    basicstyle = \ttfamily
}

\newcommand{\T}[1]{\texttt{{#1}}}
\title{Step 4 Report}
\author{刘轩奇 2018011025}
\date{2020年10月5日}
\geometry{left=2.0cm, right=2.0cm, top=2.5cm, bottom=2.5cm}
\begin{document}
    \maketitle
    \section{工作内容}
        本人选择不使用辅助工具 ANTLR 因此自己实现了 lexer 和 parser。
        \subsection{文件说明} 
            \T{montLexer.h/cpp} 词法分析器;

            \T{montParser.h/cpp} 语法分析器;

            \T{montConceiver.h/cpp} 产生中间代码;

            \T{montAssembler.h/cpp} 从中间代码产生汇编代码;
            
            \T{montLog.h} 记录编译错误信息;

            \T{montCompiler.cpp} MiniDecaf 编译器,包含主函数,编译成功返回 0 否则返回 -1,
            并将错误信息输出到 \T{std::cerr}。
        
        \subsection{本步骤完成的工作}

            \paragraph{1 词法分析} 添加了以下新的 Token:
            \begin{itemize}
                \item[*] \T{EQUAL} 等号
                \item[*] \T{NOT\_EQUAL} 不等号
                \item[*] \T{GREATER} 大于号
                \item[*] \T{GREATER\_EQUAL} 大于等于号
                \item[*] \T{LESS} 小于号
                \item[*] \T{LESS\_EQUAL} 小于等于号
                \item[*] \T{LAND} 逻辑与
                \item[*] \T{LOR} 逻辑或
            \end{itemize}

            \paragraph{2 句法分析} 修改了 \T{expression} 节点的生成逻辑,并添加了以下新的 AST 节点,对应于指导书给出的非终结符:
            \begin{itemize}
                \item[*] \T{logical\_and}
                \item[*] \T{logical\_or}
                \item[*] \T{equality}
                \item[*] \T{relational}    
            \end{itemize}

            \paragraph{3 产生中间代码} 按照指导书上的说明,将对应符号分别生成对应的
            中间代码:
            \begin{itemize}
                \item[*] \T{EQ, NEQ}
                \item[*] \T{LT, GT, LE, GE}
                \item[*] \T{LAND, LOR}
            \end{itemize}

            \paragraph{4 产生汇编代码} 新的中间代码将生成对应的汇编代码(省略了载入操作数和存储结果的语句):
            \begin{itemize}
                \item[*] \T{[EQ]} \T{sub t1,t1,t2; seqz t1,t1;}
                \item[*] \T{[NEQ]} \T{sub t1,t1,t2; snez t1,t1;}
                \item[*] \T{[LE]} \T{sgt t1,t1,t2; xori t1,t1,1;}
                \item[*] \T{[LT]} \T{slt t1,t1,t2;}
                \item[*] \T{[GE]} \T{slt t1,t1,t2; xori t1,t1,1;}
                \item[*] \T{[GT]} \T{sgt t1,t1,t2;}
                \item[*] \T{[LAND]} \T{snez t1,t1; snez t2,t2; and t1,t1,t2;}
                \item[*] \T{[LOR]} \T{or t1,t1,t2; snez t1,t1;}         
            \end{itemize}
        
    \section{思考题}
        \paragraph{1} 在表达式计算时,对于某一步运算,是否一定要先计算出所有的操作数的结果才能进行运算?
        \paragraph{答} 并非如此。例如计算逻辑或时,若首个操作数为真,则结果必为真,无需再计算第二个操作数的
        值。但在本次实验中,并未要求实现这种短路功能。
        \paragraph{2} 在 MiniDecaf 中,我们对于短路求值未做要求,但在包括 C 语言的大多数流行的
        语言中,短路求值都是被支持的。为何这一特性广受欢迎?你认为短路求值这一特性会
        给程序员带来怎样的好处?
        \paragraph{答} 一方面,短路特性能够提高程序运行的效率,当第一个操作数已经能决定
        运算结果时,不用再计算第二个操作数的值;另一方面,可以利用这种特性进行程序流程
        控制,使得程序更加简洁。
        例如希望第一个过程执行成功时不再执行第二个过程,这时就可以利用逻辑或
        的短路特性,这正是给程序员带来的便利之处。
\end{document}