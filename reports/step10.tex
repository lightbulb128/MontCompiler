\documentclass[UTF8]{ctexart}
\usepackage{graphicx}
\usepackage{amssymb}
\usepackage{amsmath}
\usepackage{subfigure}
\usepackage{geometry}
\usepackage{caption}
\usepackage{diagbox}
\usepackage{bm}
\usepackage{color}

\usepackage{enumitem}
\setitemize[1]{itemsep=0pt, partopsep=0pt, parsep=\parskip, topsep=5pt}

\usepackage{listings}
\lstdefinestyle{lfonts}{
    basicstyle = \ttfamily
}

\newcommand{\T}[1]{\texttt{{#1}}}
\title{Step 10 Report}
\author{刘轩奇 2018011025}
\date{2020年10月8日}
\geometry{left=2.0cm, right=2.0cm, top=2.5cm, bottom=2.5cm}
\begin{document}
    \maketitle
    \section{工作内容}
        本人选择不使用辅助工具 ANTLR 因此自己实现了 lexer 和 parser。
        \subsection{文件说明} 
            \T{montLexer.h/cpp} 词法分析器;

            \T{montParser.h/cpp} 语法分析器;

            \T{montConceiver.h/cpp} 产生中间代码;

            \T{montAssembler.h/cpp} 从中间代码产生汇编代码;
            
            \T{montLog.h} 记录编译错误信息;

            \T{montCompiler.cpp} MiniDecaf 编译器,包含主函数,编译成功返回 0 否则返回 -1,
            并将错误信息输出到 \T{std::cerr}。
        
        \subsection{本步骤完成的工作}

            \paragraph{1 词法分析} 没有改变。

            \paragraph{2 句法分析} 添加或修改了以下的 AST 节点:
            \begin{itemize}
                \item[*] \T{function} 内容可以包括函数和全局变量声明两种。
                \item[*] \T{globdecl} 表示全局变量声明,可以有默认初始值或指定初始值的形式。
            \end{itemize}

            \paragraph{3 产生中间代码} 在此阶段,遍历语法树过程中记录所定义的全局变量,同时查看
            是否与已经定义的全局变量或函数重名。每次遇到
            表达式中的变量调用,首先查看栈中是否包含该变量,若不包含则在全局变量中查找该变量。
            另外添加了指导书中所指示的一种中间代码,即 \T{GLOBADDR} (指导书上为 \T{GLOBALADDR})。

            \paragraph{4 产生汇编代码} 除了读取中间代码的执行代码段输出到 \T{.text} 以外,还要读取
            语义分析中所存储的全局变量,输出到 \T{.data, .bss} 段中。\T{GLOBADDR} 所生成的代码
            与实验指导书上相同,不再赘述。
        
    \section{思考题}
        \paragraph{1} 请给出将全局变量 \T{a} 的值读到寄存器 \T{t0} 所需的 riscv 指令序列。
        \paragraph{答} \T{la t0, a}
\end{document}