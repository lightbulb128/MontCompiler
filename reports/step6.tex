\documentclass[UTF8]{ctexart}
\usepackage{graphicx}
\usepackage{amssymb}
\usepackage{amsmath}
\usepackage{subfigure}
\usepackage{geometry}
\usepackage{caption}
\usepackage{diagbox}
\usepackage{bm}
\usepackage{color}

\usepackage{enumitem}
\setitemize[1]{itemsep=0pt, partopsep=0pt, parsep=\parskip, topsep=5pt}

\usepackage{listings}
\lstdefinestyle{lfonts}{
    basicstyle = \ttfamily
}

\newcommand{\T}[1]{\texttt{{#1}}}
\title{Step 6 Report}
\author{刘轩奇 2018011025}
\date{2020年10月6日}
\geometry{left=2.0cm, right=2.0cm, top=2.5cm, bottom=2.5cm}
\begin{document}
    \maketitle
    \section{工作内容}
        本人选择不使用辅助工具 ANTLR 因此自己实现了 lexer 和 parser。
        \subsection{文件说明} 
            \T{montLexer.h/cpp} 词法分析器;

            \T{montParser.h/cpp} 语法分析器;

            \T{montConceiver.h/cpp} 产生中间代码;

            \T{montAssembler.h/cpp} 从中间代码产生汇编代码;
            
            \T{montLog.h} 记录编译错误信息;

            \T{montCompiler.cpp} MiniDecaf 编译器,包含主函数,编译成功返回 0 否则返回 -1,
            并将错误信息输出到 \T{std::cerr}。
        
        \subsection{本步骤完成的工作}

            \paragraph{1 词法分析} 添加了新的 Token:
            \begin{itemize}
                \item[*] \T{QUESTION} 问号;
                \item[*] \T{COLON} 冒号;
                \item[*] \T{IF} 关键词if;
                \item[*] \T{ELSE} 关键词else。
            \end{itemize}

            \paragraph{2 句法分析} 添加了以下新的 AST 节点:
            \begin{itemize}
                \item[*] \T{conditional} 与指导书给出的相同。
                \item[*] \T{if} 从\T{statement}节点生成\T{if}节点,这与指导书直接从\T{statement}生成\T{if}语句各子节点不同。
                \item[*] \T{blockitem} 与指导书给出的相同。
            \end{itemize}

            已经实现了 \T{codeblock} 节点,该节点包括一个左大括号,零个、一个或多个\T{blockitem}和一个右括号。在 \T{statement} 
            生成中若遇到左括号则生成 \T{codeblock} 节点。

            \paragraph{3 产生中间代码} 添加了以下中间代码:
            \begin{itemize}
                \item[*] \T{BNEZ, BEQZ, BR} 与指导书给定的相同。
            \end{itemize}

            \T{LABEL} 在上一步骤中已经实现。

            \paragraph{4 产生汇编代码} 添加的中间代码生成对应的汇编代码,与指导书中相同,不再赘述。
        
    \section{思考题}
        \paragraph{1} Rust 和 Go 语言中的 if-else 语法与 C 语言中略有不同,
        它们都要求两个分支必须用大括号包裹起来,而且条件表达式不需要用括号包裹起来:
        
        \noindent\rule{\textwidth}{1pt}
        \begin{lstlisting}[style=lfonts]
if CONDITION {
  // execute when CONDITION is true
} else {
  // execute when CONDITION is false
}
        \end{lstlisting}
        \noindent\rule{\textwidth}{1pt}

        请问相比 C 的语法,这两种语言的语法有什么优点?

        \paragraph{答} 一方面,不用小括号包裹条件显得简洁且自然;另一方面,分支必须用大括号
        包裹,则不再需要区分变量定义与其他语句,它们均可以放在大括号中,程序也显得比较规整。
\end{document}