\documentclass[UTF8]{ctexart}
\usepackage{graphicx}
\usepackage{amssymb}
\usepackage{amsmath}
\usepackage{subfigure}
\usepackage{geometry}
\usepackage{caption}
\usepackage{diagbox}
\usepackage{bm}
\usepackage{color}

\usepackage{listings}
\lstdefinestyle{lfonts}{
    basicstyle = \ttfamily
}

\newcommand{\T}[1]{\texttt{{#1}}}
\title{Step 1 Report}
\author{刘轩奇 2018011025}
\date{2020年9月25日}
\geometry{left=2.0cm, right=2.0cm, top=2.5cm, bottom=2.5cm}
\begin{document}
    \maketitle
    \section{工作内容}
        本人选择不使用辅助工具 ANTLR 因此自己实现了 lexer 和 parser。
        \subsection{文件说明} 
            \T{montLexer.h/cpp} 词法分析器;

            \T{montParser.h/cpp} 语法分析器;

            \T{montConceiver.h/cpp} 产生中间代码;

            \T{montAssembler.h/cpp} 从中间代码产生汇编代码;
            
            \T{montLog.h} 记录编译错误信息;

            \T{montCompiler.cpp} MiniDecaf 编译器,包含主函数,编译成功返回 0 否则返回 -1,
            并将错误信息输出到 \T{std::cerr}。
        
        \subsection{本步骤完成的工作}

            \paragraph{1 词法分析} 在 \T{MontLexer} 中,定义了 Token 的分类,所有不同的关键词
            /保留字被看作不同的 Token,这样可以减少后续语法分析的工作量。其余 Token 类型还包括标
            识符 Identifier, 各种符号,整数值,字符值等。\T{MontLexer} 运行过程类似于有限状态
            自动机,每读入一个字符后,判断当前读入的这个 Token 是否有可能是符号、标识符、关键字、
            常数值中的任意一种,并判断是否已经读完一整个 Token ,若是,则返回此 Token 
            作为 nextToken()。在这一步中也判断了常数值过大的词法错误。

            \paragraph{2 句法分析} 在 \T{MontParser} 中,只定义了两种树节点: 
            \T{MontNode} 和 \T{MontTokenNode},分别表示内部节点和终结符节点。所有的
            内部节点由其类型标识,例如函数节点、语句节点、表达式节点等,有至少一个
            子节点;而终结符节点包含且仅包含一个 Token,不含子节点。抽象语法树
            生成过程中,调用各种 \T{tryParse...()} 方法来产生对应的树节点。例如:
            对于产生式

            \centerline{function : type Identifier LParen RParen codeblock}

            \noindent 将会分别调用 \T{tryParseType()}, \T{tryParse(TK\_IDENTIFIER)}, 
            \T{tryParse(TK\_LPAREN)}, \T{tryParse(TK\_RPAREN)}, 
            \T{tryParseCodeblock()} 来生成子节点。

            \paragraph{3 产生中间代码} 在 \T{MontConceiver} 中,
            按照指导书上的说明,将 \T{value} 节点生成 
            \T{PUSH} 语句(根据产生式 \T{expression} : \T{value}),将 \T{return} 
            节点生成 \T{RET} 语句。在这一步中也检查了函数名是否为 \T{main}。

            \paragraph{4 产生汇编代码} 在 \T{MontAssembler} 中,按照指导书的
            说明,将对应中间代码转换为对应汇编语句,不再赘述。

            \paragraph{5 编译错误行号追踪} 在词法分析、语法分析和生成中间代码
            的过程中,如若遇到编译错误,则在 \T{std::cerr} 中输出编译错误信息。
            同时,由于在 lexer 读入和 Token 结构中记录了行列号,则输出编译错误信息
            时也可以同时输出,便于用户查错。

            \paragraph{6 尚未完整实现或当前尚未使用的功能} 支持以 \T{0x...} 十六进制
            形式 读入整数常量;支持读入 \T{char} 类型 作为常量,虽然存储上依然按照整数
            (即ASCII编码)存储,支持普通显示字符(例如 \T{'A'})、转义字符(例如 \T{'\\A'})、
            八进制表示转义字符(例如 \T{'\\123'})、十六进制转义字符(例如 \T{'\\x41'});
            在 lexer 和 parser 阶段支持单变量定义语句(例如 \T{int a;})。
        
    \section{思考题}
        \paragraph{1} 修改 \T{minilexer} 的输入(\T{lexer.setInput} 的参数),使得 lex 报错,给出一个简短的例子。
        \paragraph{答} 只需要出现非法字符即可。例如 \T{int main()\{return 一百二十三;\}}
        
        \paragraph{2} 修改 \T{minilexer} 的输入,使得 lex 不报错但 parse 报错,给出一个简短的例子。
        \paragraph{答} 只需要出现语法错误即可。例如 \T{int main()\{turn 123;\}}

        \paragraph{3} 在 riscv 中,哪个寄存器是用来存储函数返回值的?
        \paragraph{答} \T{a0} 寄存器。
\end{document}