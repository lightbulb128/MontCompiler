\documentclass[UTF8]{ctexart}
\usepackage{graphicx}
\usepackage{amssymb}
\usepackage{amsmath}
\usepackage{subfigure}
\usepackage{geometry}
\usepackage{caption}
\usepackage{diagbox}
\usepackage{bm}
\usepackage{color}

\usepackage{listings}
\lstdefinestyle{lfonts}{
    basicstyle = \ttfamily
}

\newcommand{\T}[1]{\texttt{{#1}}}
\title{Step 2 Report}
\author{刘轩奇 2018011025}
\date{2020年9月28日}
\geometry{left=2.0cm, right=2.0cm, top=2.5cm, bottom=2.5cm}
\begin{document}
    \maketitle
    \section{工作内容}
        本人选择不使用辅助工具 ANTLR 因此自己实现了 lexer 和 parser。
        \subsection{文件说明} 
            \T{montLexer.h/cpp} 词法分析器;

            \T{montParser.h/cpp} 语法分析器;

            \T{montConceiver.h/cpp} 产生中间代码;

            \T{montAssembler.h/cpp} 从中间代码产生汇编代码;
            
            \T{montLog.h} 记录编译错误信息;

            \T{montCompiler.cpp} MiniDecaf 编译器,包含主函数,编译成功返回 0 否则返回 -1,
            并将错误信息输出到 \T{std::cerr}。
        
        \subsection{本步骤完成的工作}

            \paragraph{1 词法分析} 添加了三种 Token,类型分别为 \T{Exclamation} 感叹号,
            \T{Tilde} 波浪线符号和 \T{Minus} 负号。

            \paragraph{2 句法分析} 添加了 \T{Unary} 类型节点,生成此节点时,查看 lexer
            提供的第一个符号是否是一元运算符,若是则继续递归生成 \T{Unary} 否则生成
            \T{Value} 类型节点。

            \paragraph{3 产生中间代码} 
            按照指导书上的说明,将三种符号对应生成三种中间代码:逻辑非即感叹号对应
            \T{LNOT} 操作,按位取反即波浪线符号对应 \T{NOT} 操作,负数对应 \T{NEG} 操作。

            \paragraph{4 产生汇编代码} 在 \T{MontAssembler} 中,按照指导书的
            说明,将对应中间代码转换为对应汇编语句,逻辑非即感叹号对应
            \T{seqz} 操作,按位取反即波浪线符号对应 \T{not} 操作,负数对应 \T{neg} 操作。
        
    \section{思考题}
        \paragraph{1} 我们在语义规范中规定整数运算越界是未定义行为,运算越界可以简单理解
        成理论上的运算结果没有办法保存在32位整数的空间中,必须截断高于32位的内容
        。请设计一个表达式,只使用 \T{-\textasciitilde!} 这三个单目运算符和 $[0, 2^{31} - 1]$ 范围
        内的非负整数,使得运算过程中发生越界。
        \paragraph{答} 取 \T{-\textasciitilde 2147483647} 即可。

\end{document}